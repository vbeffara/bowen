\section{Unicité de la mesure de Gibbs}

  Dans un premier temps, on introduit quelques notations.

  \begin{definition}
    \label{def:meas_set}
    \leanok
    On note $\Ms{n}$ l'ensemble des mesures de probabilité sur $\s_n$, et $\Mss{n}$ le sous-ensemble de $\Ms{n}$
    constitué des mesures $\sigma$-invariantes.
  \end{definition}

  \begin{definition}
    \label{def:gibbs_meas}
    \uses{def:meas_set, def:shift}
    \leanok
    \lean{IsGibbs}
    Soit $\phi \in \Cs{n}$ et $\mu \in \Ms{n}$.
    On dit que $\mu$ est une mesure de Gibbs pour un potentiel $\phi$ si $\mu \in \Mss{n}$ et s'il existe $P(\phi) \in \R$ et $c_1, c_2 > 0$ tels qu'on ait
    $$\frac{\mu(C(x, m))}{\exp(-Pm + \sum_{k=1}^{m-1}{\phi(\sigma^k(x))})} \in [c_1, c_2]$$
    pour tout $x \in \s_n$ et $m \in \N$.
  \end{definition}

  Dans cette section, on montre la partie concernant l'unicité du théorème suivant.

  \begin{theorem}
    \label{thm:gibbs}
    \uses{def:holder_fn_set, def:gibbs_meas, thm:existence, prop:mu_mixing, prop:unique, lem:fn_equiv_pos_coords}
    Soit $\phi \in \mathcal H(\s_n)$, alors il existe une unique mesure de Gibbs $\mu_{\phi} = \mu \in \mathcal M_{\sigma}(\s_n)$
  \end{theorem}

  Dans les sections suivantes, on construira une mesure de Gibbs mélangeante notamment grâce au théorème de Ruelle.
  C'est l'ergodicité de cette mesure qui entrainera l'unicité de la mesure de Gibbs.

  \begin{definition}
    \label{def:ergodique}
    \uses{def:meas_set, def:shift}
    Soit $\mu \in \Ms{n}$.
    On dit que $\mu$ est ergodique (par rapport à $\sigma$) si pour tout borélien $E \in \mathcal{B}(\s_n)$ tels que $\sigma^{-1}E=E$, on a
	$$\mu(E) = 0 \text{ ou } \mu(E^c) = 0,$$
  \end{definition}

  \begin{definition}
    \label{def:mixing}
    \uses{def:meas_set, def:shift}
    Soit $\mu \in \Ms{n}$. On dit que $\mu$ est mélangeante (par rapport à $\sigma$) si pour tout borélien $E, F \in \mathcal{B}(\s_n)$, on a
	$$\mu(E\cap\sigma^{-n}F) \underset{n\to\infty}{\longrightarrow} \mu(E)\mu(F).$$
  \end{definition}

  \begin{remark}
    On a l'implication "mélangeante" $\implies$ "ergodique".
  \end{remark}

  \begin{lemma}
    \label{lem:sig_inv_imp_cst}
    \uses{def:ergodique}
    Soit $f \colon \s_n \longrightarrow \R$ intégrable par rapport à une mesure $\mu \in \Mss{n}$ ergodique.
    Supposons que $f \circ \sigma = f$ $\mu$-presque partout.
    Alors $f$ est constante $\mu$-presque partout.
  \end{lemma}

  \begin{proof}
    \proves{lem:sig_inv_imp_cst}
    Pour montrer ce lemme on considére les ensembles $E_c = f^{-1}(\{c\})$ pour tout $c\in\R$.
    Comme $f \circ \sigma = f$ $\mu$-p.p., on a $\sigma^{-1}E_c = E_c$, et donc par ergodicité de $\mu$,
    $$\mu(E_c) = 0 \text{ ou } \mu(E_c) = 1.$$
    Il est donc clair qu'il existe au plus un $c\in\R$ tel que $f = c$ pour $\mu$-presque tout $x \in \s_n$.
  \end{proof}

  \begin{proposition}
    \label{prop:unique}
    \uses{def:ergodique, def:gibbs_meas}
    Supposons qu'il existe une mesure de Gibbs $\mu$ ergodique pour $\phi \in \Cs{n}$.
    Alors cette mesure est l'unique mesure de Gibbs associé à $\phi$.
  \end{proposition}

  \begin{proof}
    \proves{prop:unique}
    \uses{lem:sig_inv_imp_cst, prop:boules} % + Radon-nikodym & régularité extérieure
    Soit $\mu, \mu' \in \Msp{n}$ deux mesures de Gibbs avec $\mu$ ergodique, $c_1, c_1', c_2, c_2' > 0$ et $P, P' \in \R$
    des constantes telles que pour tout $m \in \N$ et $x \in \s_n$ on ait
    $$\left\{\begin{array}{lll}
	c_1  &\leq \frac{\mu(C(x,m))}{\exp{(-Pm + S_m\phi(x))}} &\leq c_2 \vspace{1em} \\
	c_1' &\leq \frac{\mu'(C(x,m))}{\exp{(-P'm + S_m\phi(x))}} &\leq c_2'
      \end{array}\right.$$

      D'abord, montrons que $P = P'$.
      Soit $m \in \N$ et $T_m = \left\{x \in \s_n \mid x_i = 1 \text{ si } i \not\in \llbracket 0, m-1 \rrbracket \right\}$,
      de telle sorte que $\s_n = \bigsqcup_{x\in T_m}{C(x,m)}$, et $T_m$ est fini.
      On a alors
      $$c_1'e^{-P'm}\sum_{x\in T_m}{e^{S_m\phi(x)}} \leq \sum_{x\in T_m}{\mu'(C(x,m))}
            = 1 = \sum_{x\in T_m}{\mu'(C(x, m))} \leq c_2'e^{-P'm}\sum_{x\in T_m}{e^{S_m\phi(x)}},$$
      et donc
      $$P - \frac 1 m \log{c_2'} \leq \frac 1 m \log{\sum_{x\in T_m}{e^{S_m\phi(x)}}} \leq P - \frac 1 m \log{c_1'}.$$
      Par le théorème des gendarmes, on trouve donc $P' = \lim_{m\to\infty}{\frac 1 m\sum_{x\in T_m}{e^{S_m\phi(x)}}}$.
      Grâce au même raisonnement, on en conclut que $P = P'$ car ils sont tout deux égaux à la même limite.

      Grâce aux estimations sur $\mu$ et $\mu'$ sur les cylindres, on a alors pour tout $x \in \s_n$ et $m \in \N$,
      $$\mu'(C(x, m)) \leq \frac{c_2'}{c_1}\mu(C(x, m)).$$
      Comme $\mu$ et $\mu'$ sont invariantes par $\sigma$, on peut étendre le résultat sur les ensembles de la forme
      $\left\{y\in\s_n \mid \forall i \in \llbracket -m, m\rrbracket, x_i = y_i\right\}$ pour tout $x \in \s_n$ et $m \in \N$,
      c'est-à-dire sur une base de la topologie de $\s_n$.
      De plus, comme tout ouvert de $\s_n$ est une union disjointe dénombrable de ces éléments d'après la proposition \ref{prop:boules},
      on peut alors étendre cette inégalité aux ouverts de $\s_n$.
      Par régularité extérieure par rapport aux ouverts, on a pour tout borélien $B \in \Bo{\s_n}$,
      $$\mu'(B) = \inf\left\{\mu'(O) \mid O \text{ ouvert }, B \subseteq O\right\}.$$
      De là, on a pour tout ouvert $B \subseteq O$,
      $$\mu'(B) \leq \mu'(O) \leq \frac{c'_2}{c_1} \mu(O),$$
      et donc $\mu'(B) \leq \frac{c'_2}{c_1}\inf\{\mu(O) \mid O \text{ ouvert }, B \subseteq O\} = \frac{c_2'}{c_1}\mu(B)$.
      Donc pour tout borélien $B \in \Bo{\s_n}$ on a $\mu'(B) \leq \frac{c_2'}{c_1}\mu(B)$, ainsi $\mu'$ est absolument continue par rapport à $\mu$.
      D'après le théorème de Radon-Nikodym, $\mu'$ admet une densité $f$ par rapport à $\mu$.
      En appliquant $\sigma$, on obtient
      $$\mu' = \sigma_*\mu' = (f\circ\sigma^{-1})\cdot\sigma_*\mu = (f\circ\sigma^{-1})\cdot\mu.$$
      Par unicité de la dérivée de Radon-Nikodym, $f = f \circ \sigma^{-1}$ $\mu$-presque partout.
      Or comme $\mu$ est ergodique par hypothèse, alors il existe une constante $c \in \R$ tel que $f = c$ $\mu$-presque partout.
      Finalement,
      $$1 = \mu'(\s_n) = \int_{\s_n}{c\,d\mu} = c.$$
      Donc $\mu = \mu'$, ce qui prouve l'unicité.
  \end{proof}
